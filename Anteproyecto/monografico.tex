\documentclass[titlepage, 12pt, a4paper, oneside]{article}
\usepackage[utf8]{inputenc}
\usepackage[spanish, es-tabla]{babel}
\usepackage[T1]{fontenc}
\usepackage{hyperref}
\usepackage[numbib]{tocbibind}
\usepackage{tikz}
\usepackage[top=1in, bottom=1.25in, left=1.25in, right=1.25in]{geometry}
\usepackage{xcolor}

\usepackage{fancyhdr}
\pagestyle{fancy}
\fancyhf{}
\rhead{\textit{\color[rgb]{0.0,0.424,0.616}Nombre del estudiante}}
\lhead{}
\rfoot{}
\renewcommand{\headrulewidth}{0pt}

\usepackage{pgfgantt}

\title{}
\date{}
\renewcommand{\familydefault}{\sfdefault}

\begin{document}
\thispagestyle{empty}
\tikz[remember picture,overlay] \node[opacity=1.0,inner sep=0pt] at (current page.center){\includegraphics[width=\paperwidth,height=\paperheight]{Plantilla_AnteProyectoTFG-portada}};

\begin{center}
  \vspace{4cm}
  {\color{white} \Huge \textbf{Título del Trabajo}}
\end{center}

\Large

\vspace{16.5ex}
\begin{tabular}{ll}
  ~~~~~~~~~~~~~~~~~ & Nombre del estudiante
\end{tabular}

\vspace{1.2cm}
\begin{tabular}{ll}
  ~~~~~~~~~~~~~~~~~ & Grado en Ingeniería Informática
\end{tabular}

\vspace{1.1cm}
\begin{tabular}{ll}
  ~~~~~~~~~~~~~~~~~ & Vicente González Ruiz
\end{tabular}

\vspace{1.2cm}
\begin{tabular}{ll}
  ~~~~~~~~~~~~~~~~~ & Departamento de Informática
\end{tabular}

\vspace{0.95cm}
\begin{tabular}{ll}
  ~~~~~~~~~~~~~~~~~ & Trabajo Monográfico
\end{tabular}

\vspace{0.95cm}
\begin{tabular}{ll}
  ~~~~~~~~~~~~~~~~~ & Bla, bla, bla
\end{tabular}

\clearpage

\tikz[remember picture,overlay] \node[opacity=1.0,inner sep=0pt] at (current page.center){\includegraphics[width=\paperwidth,height=\paperheight]{Plantilla_AnteProyectoTFG-paginas}};

\normalsize

\section{Introducción}
En esta sección se comentarán las características del problema a resolver y los antecedentes del TFG. Las referencias bibliográficas deberían citarse de esta manera \cite{einstein1922kosmologische}.

\section{Interés y objetivos}
En esta sección se comentará el objetivo principal a conseguir y el por qué de estos (justificación).

\section{Fases de desarrollo}
En esta sección se comentarán las distintas fases en las que se ha
dividido el trabajo a realizar.

A modo de ejemplo, se contemplan las siguientes unidades de trabajo y
temporizaciones (véase la Figura~\ref{fig:temporizacion}):
\begin{enumerate}
  \item {ANÁLISIS:} Estudio y análisis de la solución actual y de
    otras soluciones similares, incluyendo las bibliotecas existentes
    (50 h).
  \item {IDENTIFICACIÓN}: Identificación de los requerimientos (50 h).
  \item {APROVISIONAMIENTO}: Instalación y configuración del entorno
    de desarrollo (50h).
  \item {IMPLEMENTACIÓN}: Implementación de la solución (50h).
  \item {EVALUACIÓN}: Evaluación de la solución, haciendo énfasis en
    el grado de consecución de los objetivos (20h).
  \item {REDACCIÓN}: Redacción de la memoria (30h).
\end{enumerate}

\begin{figure}
  \begin{center}
    \resizebox{\columnwidth}{!}{
      \begin{ganttchart}{1}{30}{10}
        \gantttitle{Bloques de 10 horas}{30} \\
        \gantttitlelist{1,...,30}{1} \\
        \ganttbar{ANÁLISIS}{1}{5} \\ % 50h
        \ganttlinkedbar{IDENTIFICACIÓN}{6}{10} \\ % 50 h
        \ganttlinkedbar{APROVISIONAMIENTO}{11}{15} \\ % 50 h
        \ganttlinkedbar{IMPLEMENTACiÓN}{16}{20} \\ % 50 h
        \ganttlinkedbar{EVALUACIÓN}{21}{25} \\ % 50 h
        \ganttlinkedbar{REDACCIÓN}{26}{30} % 50 h
      \end{ganttchart}
    }
  \end{center}
  \caption{Temporización del TFG.\label{fig:temporizacion}}
\end{figure}

%\section{Revisión bibliográfica}
%Pues eso. Cuando sea posible, usar referencias contrastadas (artículos y libros).

\section{Materiales y métodos}
En esta sección se comentarán los diferentes materiales o herramientas
y métodos o técnicas necesarias para la realización del TFG.

\section{Resultados esperados}
Los trabajos monográficos podrían perseguir objetivos relativamente
inciertos, hipótesis a verificar, etc. Por tanto, en esta sección se
realizará una serie de experimentos que demuestren que los objetivos
se cumplen y en qué grado. Los resultados serán discutidos.

\section{Conclusiones esperadas}
Un breve resumen sobre el proyecto desarrollado.

\bibliographystyle{plain}
\bibliography{../bibliografia}

\begin{center}
  \textbf{Firma del director (codirector)}
\end{center}

\end{document}
